\input{style.tex}

\title{Lab 02 report}
\author{Sergey}

\date{\today}

\begin{document}
\thispagestyle{empty}

\noindent \begin{minipage}{0.15\textwidth}
	\includegraphics[width=\linewidth]{b_logo}
\end{minipage}
\noindent\begin{minipage}{0.85\textwidth}\centering
	\textbf{Министерство науки и высшего образования Российской Федерации}\\
	\textbf{Федеральное государственное бюджетное образовательное учреждение высшего образования}\\
	\textbf{«Московский государственный технический университет имени Н.Э.~Баумана}\\
	\textbf{(национальный исследовательский университет)»}\\
	\textbf{(МГТУ им. Н.Э.~Баумана)}
\end{minipage}

\noindent\rule{16cm}{3pt}
\newline\newline
\noindent ФАКУЛЬТЕТ $\underline{\text{«Информатика и системы управления»}}$ \newline\newline
\noindent КАФЕДРА $\underline{\text{«Программное обеспечение ЭВМ и информационные технологии»}}$\newline


\begin{center}
	\noindent\begin{minipage}{1.3\textwidth}\centering
	\Large\textbf{   ~~~ Лабораторная работа №4}\newline
	\textbf{по дисциплине "Анализ Алгоритмов"}\newline\newline\newline
	\end{minipage}
\end{center}

\noindent\textbf{Тема} $\underline{\text{Параллельное сложение матриц}}$\newline\newline
\noindent\textbf{Студент} $\underline{\text{Сабуров С. М.}}$\newline\newline
\noindent\textbf{Группа} $\underline{\text{ИУ7-53Б}}$\newline\newline
\noindent\textbf{Преподаватель} $\underline{\text{Волкова Л. Л.}}$\newline

\begin{center}
	\mbox{}
	\vfill
	Москва
\end{center}

\begin{center}
	\the\year ~г.
\end{center}
\clearpage

\renewcommand\contentsname{\hfill{\normalfont{СОДЕРЖАНИЕ}}\hfill}  %Оглавление
\tableofcontents
\newpage

\anonsection{Введение}


Многопоточность — способность центрального процессора (ЦПУ) или
одного ядра в многоядерном процессоре одновременно выполнять несколько процессов или потоков, соответствующим образом поддерживаемых операционной системой.
Многопоточная парадигма стала более популярной с конца 1990-х годов,
поскольку усилия по дальнейшему использованию параллелизма на уровне
инструкций застопорились.
Смысл многопоточности — квазимногозадачность на уровне одного исполняемого процесса.
Значит, все потоки процесса помимо общего адресного пространства
имеют и общие дескрипторы файлов. Выполняющийся процесс имеет как
минимум один (главный) поток.




Целью данной работы является реализация и анализ алгоритмов параллельного вычисления на примере сложения матриц, .



Для достижения поставленной цели необходимо выполнить следующие задачи:
\begin{itemize}
    \item изучить понятие параллельных вычислений;
    \item привести схемы классического и параллельного сложения матриц.
    \item реализовать классический алгоритм сложения матриц;
    \item реализовать параллельный алгоритм сложения матриц;
    \item сравнить их временные характеристики экспериментально;
    \item на основании проделанной работы сделать выводы.
\end{itemize}

\section{Аналитическая часть}

В данном разделе будут рассмотрены алгоритмы сложения матриц

\subsection{Стандартный алгоритм}

Пусть даны две прямоугольные матрицы

\begin{equation}
    A =
      \begin{pmatrix}
        a_{11} & a_{12} & \cdots & a_{1m} \\
        a_{21} & a_{22} & \cdots & a_{2m} \\
        \vdots & \vdots & \ddots & \vdots \\
        a_{l1} & a_{l2} & \cdots & a_{lm}
      \end{pmatrix},
\end{equation}

\begin{equation}
    B =
    \begin{pmatrix}
      b_{11} & b_{12} & \cdots & b_{1m} \\
      b_{21} & b_{22} & \cdots & b_{2m} \\
      \vdots & \vdots & \ddots & \vdots \\
      b_{l1} & b_{l2} & \cdots & b_{lm}
    \end{pmatrix}.
\end{equation}

Тогда матрица $C$ размерностью $l \times m$

\begin{equation}
    C =
      \begin{pmatrix}
        c_{11} & c_{12} & \cdots & c_{1m} \\
        c_{21} & c_{22} & \cdots & c_{2m} \\
        \vdots & \vdots & \ddots & \vdots \\
        c_{l1} & c_{l2} & \cdots & c_{lm}
      \end{pmatrix},
\end{equation}

в которой:

\begin{equation}
    \displaystyle
    c_{ij} = a_{ij} + b_{ij}\quad (i = \overline{1, l}; j = \overline{1, m} )
\end{equation}

будет называться сложением матриц $A$ и $B$. Стандартный алгоритм
реализует данную формулу.

\subsection{Параллельный алгоритм}

Поскольку все элементы матрицы 𝐶 вычисляется независимо друг от друга  и матрицы 𝐴 и
𝐵 остаются неизменными, для параллельного вычисления произведения достаточно 
распределить вычисление элементов матрицы 𝐶 между потоками.
Поскольку мы имеем некие аппаратные ограничения, производить данные вычисления для каждого
элемента результирующей матрицы в отдельности не эффективно. Следовательно, данная проблема решается разделением  элементов результирющей матрицы по строкам и параллельным вычислением результатов для каждого из разделов.
\subsection{Вывод}

Были рассмотрены алгоритмы классического сложения матриц и параллельного.Поскольку в стандартном алгоритме сложения матриц элементы результирующей матрицы вычисляются независимо друг от друга, есть возможность  реализовать параллельный  алгоритма.
\begin{itemize}
	\item Входные данные : количество строк в первой матрице, количество столбцов в первой матрице, элементы первой матрицы, Количество строк во второй матрице, количество столбцов во второй матрице, элементы второй матрицы.
	\item Выходные данные : на выходе имеем матрицу - результат сложения двух матриц, являющихся входными данными.
	\item Ограничения, в рамках которых будет работать программа : размеры матриц должны быть целыми положительными числами, элементы матриц должные быть также числами(допустим вещественный тип).
	\item Функциональные требования : функции, представленные на листингах 2 - 3 должны вычислять результат умножения двух матриц.
	\item Требования к программному обеспечению : к программе предъявляется ряд требований:
			\begin{itemize}
			    \item на вход подаются размеры матриц (натуральные числа) и самы матрицы, которые нужно сложить;
			    \item на выходе - результаты сложения матриц алгоритмами простого сложения матриц, параллельного сложения матриц.
			\end{itemize}
\end{itemize}

\section{Конструкторская часть}
В данном разделе будут приведены схемы  алгоритмов, описание структур данных, способы тестирования и классы эквивалентности.

\subsection{Разработка алгоритмов}

На рисунках \ref{img:classic}-\ref{img:parallel} приведены схемы алгоритмов простого сложения матриц и параллельного сложения матриц соответственно.

\subsection{Описание структур данных}
Был реализован класс Matrix, объединивший в себе алгоритмы работы с матрицей и элементы матрицы. Данный класс состоит из массива указателей на строки хранимой матрицы, операции ввода-вывода, а так же метод сложения матриц. Ко всему прочему, в данном классе присутствуют конструкторы и деструктор. На рисунке \ref{img:diag} изображена диаграмма класса Matrix.


\subsection{Способы тестирования и классы эквивалентности}
Была выбрана методика тестирования черным ящиком. 
Классы эквивалентности:
\begin{itemize}
	\item Матрицы одинкаовых размеров.
	\item Количество столбцов первой матрицы равно количеству строк матрицы, при этом матрицы не одинаковых размеров.
	\item Матрицы представляют собой 1 элемент.
	\item Количество столбцов первой матрицы не равно количеству строк матрицы.
\end{itemize}
\subsection{Вывод}

На основе теоретических данных, полученных из аналитического раздела, были построены схемы требуемых алгоритмов, описаны стртуктуры данных, выделены способы тестирования и классы эквивалентности.

\begin{figure}
    \centering
    \includegraphics[scale=0.75]{classic.jpg}
    \caption{Схема алгоритма простого сложения матриц}
    \label{img:classic}
\end{figure}

\begin{figure}
    \centering
    \includegraphics[scale=0.75]{parallel.jpg}
    \caption{Схема алгоритма параллельного сложения матриц}
    \label{img:parallel}
\end{figure}

\begin{figure}
    \centering
    \includegraphics[scale=0.75]{diag_class.png}
    \caption{Диаграмма класса Matrix.}
    \label{img:diag}
\end{figure}

\section{Технологическая часть}

В данном разделе приведены требования к программному обеспечению, средства реализации и листинги кода.



\subsection{Средства реализации}


Для реализации программ был выбран язык программирования C++ [1]. Данный язык был выбран потому, что в нем присутствует инструментарий для замера процессорного времени и тестирования.

\subsection{Листинги кода}
\begin{lstlisting}[caption=Описание класса Matrix, label=list:canon, language={}]
template<class T>
class Matrix
{
public:
	int M;
	int N;
	vector<vector<T>> data;
	Matrix();
	Matrix(const Matrix<T>& M_);

	Matrix(int M_, int N_);


	Matrix<T> classic_sum(  Matrix<T>& M2);

	~Matrix<T>();

	Matrix<T>& input_matrix();

	void output_matrix();

};
\end{lstlisting}

\begin{lstlisting}[caption=Алгоритм простого сложения матриц, label=list:canon, language={}]
template <typename T>
Matrix<T> Matrix<T>::classic_sum( Matrix<T>& M2)
{
	Matrix<T> res(this->M, M2.N);
	for (int i = 0; i < res.M; i++)
	{
		for (int j = 0; j < res.N; j++)
		{
			res.data[i][j] = this->data[i][j] + M2.data[i][j];
		}
	}
	return res;

}
\end{lstlisting}

\begin{lstlisting}[caption=Алгоритм параллельного сложения матриц, label=list:vinograd, language={}]
template<class T>
void parallel_add(Matrix<T> matrix1, Matrix<T> matrix2, Matrix<T> &result, int thread, int threads_amount)
{

	for (int i = thread; i < result.M; i += threads_amount)
	{
		for (int j = 0; j < result.N; j++)
		{
			result.data[i] [j] = (matrix1.data[i][j] + matrix2.data[i][j]);
		}

	}
}
template<class T>
Matrix<T>  parallel_sum( Matrix<T> M1,  Matrix<T> M2, int threads_amount)
{
	Matrix<T> res(M1.M, M2.N);

	vector<thread> threads(threads_amount);

	for (int thread_ = 0; thread_ < threads_amount; thread_++)
	{
		threads[thread_] = thread(parallel_add<T>, M1, M2, std::ref(res), thread_, threads_amount);
	}

	for (int i = 0; i < threads_amount; i++)
	{
		threads[i].join();
	}

	return res;
}

\end{lstlisting}
\subsection{Тестирование функций}

В таблице \ref{tab:tests} приведены модульные тесты для функций сложения матриц выше перечисленными методами. Все тесты были пройдены успешно. \\

\begin{table}[hb]
    \caption{\centering Тестирование функций сложения матриц}
    \centering
    \begin{tabular}{ccc}
    Матрица 1 & Матрица 2 & Ожидаемый результат \\ \hline
    $\begin{pmatrix}
        1 & 2 & 3 \\
        4 & 5 & 6 \\
        7 & 8 & 9
    \end{pmatrix}$
    &$\begin{pmatrix}
        1 & 2 & 3 \\
        4 & 5 & 6 \\
        7 & 8 & 9
    \end{pmatrix}$
    &$\begin{pmatrix}
        2 & 4 & 6 \\
        8 & 10 & 12 \\
        14 & 16 & 18
    \end{pmatrix}$\\
    $\begin{pmatrix}
        1 & 2 \\
        4 & 5
    \end{pmatrix}$
    &$\begin{pmatrix}
        1 & 2 \\
        4 & 5
    \end{pmatrix}$
    &$\begin{pmatrix}
        2 & 4 \\
        8 & 10
    \end{pmatrix}$\\
    $\begin{pmatrix}
        8
    \end{pmatrix}$
    &$\begin{pmatrix}
        4
    \end{pmatrix}$
    &$\begin{pmatrix}
        12
    \end{pmatrix}$\\
    $\begin{pmatrix} 1 & 2 \end{pmatrix}$ & $\begin{pmatrix} 3 \end{pmatrix}$ & Сложение невозможно
    \end{tabular}
    \label{tab:tests}
    \end{table}

\subsection{Вывод}

Были разработаны и протестированы реализации алгоритмов: простой алгоритм сложения матриц, параллельный алгоритм сложения матриц.

\section{Исследовательская часть}
В данном разделе будут приведены результаты исследовательской деятельности - замеры процессорного времени работы алгоритмов и тестирование алгоритмов.

\subsection{Технические характеристики}

Технические характеристики электронно-вычислительнй машины, на которой выполнялось тестирование:

\begin{itemize}
    \item операционная система: Windows 10 64-bit;
    \item оперативная память: 8 гигабайт ;
    \item процессор: Intel i5 7th gen.
\end{itemize}


Тестирование проводилось на ноутбуке, включенном в сеть электропитания. Во время тестирования ноутбук был нагружен только встроенными приложениями окружения рабочего стола, окружением рабочего стола, а также непосредственно системой тестирования.

\subsection{Время выполнения алгоритмов}

Был проведен замер времени работы каждого из алгоритмов с помощью функции std::chrono::system clock::now. Эта функция замеряет процессорное время выполнения функции и усредняет его (проводится 20 замеров). В таблице \ref{tab:time_best} содержатся результаты исследований.

На рисунках \ref{img:plot_best}, \ref{img:plot_worst} демонстрируется зависимость времени выполнения конкретных реалзиаций алгоритмов сложения матриц от размера стороны квадратной матрицы и количества потоков соответственно. \\

\begin{table}[ht]
    \caption{\centering Время выполнения реализаций алгоритмов (в секундах) при количестве потоков 4.}
    \centering
    \begin{tabular}{|c|c|c|c|}
    \hline
    Размер & К      & П     \\ \hline
    100    & 0.021534 & 0.020617 \\ \hline
    200    & 0.040475 & 0.039583  \\ \hline
    500    & 0.518521  & 0.197411  \\ \hline
    1000    & 2.053641  & 0.519283  \\ \hline
 
    \end{tabular}
    \label{tab:time_best}
\end{table}

\begin{figure}
    \centering
    \includegraphics[scale=0.65]{ths.png}
    \caption{Зависимость времени выполнения алгоритмов от количества потоков при размере матрицы 1000 на 1000}
    \label{img:plot_best}
\end{figure}



\begin{figure}
    \centering
    \includegraphics[scale=0.65]{sizes.png}
    \caption{Зависимость времени выполнения алгоритмов от размера стороны квадратной матрицы при количестве потоков 4}
    \label{img:plot_worst}
\end{figure}

\subsection{Вывод}

В данном разделе было произведено сравнение количества затраченного времени вышеизложенных алгоритмов.
Наиболее эффективным по времени алгоритмом при работе с матрицами больших размерностей (более 200 элементов) оказался параллельный алгоритм, работащий на 8 потоках.
При работе с матрицами малых размерностей (менее 200 элементов) стандартный алгоритм оказался более эффективным (~ в 4 раза в сравнении с 16 потоками), что связано с дополнительным затратами, которые необходимы при реализации многопоточности (создание потоков, реализация совместного доступа к ресурсам).


\anonsection{Заключение}


В ходе выполнения работы была достигнута цель выполнены все поставленные задачи:

\begin{itemize}
    \item реализовать классический алгоритм сложения матриц;
    \item реализовать параллельный алгоритм сложения матриц;
    \item сравнить их временные характеристики экспериментально;
    \item на основании проделанной работы сделать выводы.
\end{itemize}

Экспериментально были установлены различия в производительности различных алгоритмов сложения матриц. Параллельный алгоритм  имеет большую эффективность(при размерностях выше 200), нежели классический алгоритм сложения матриц. 
\anonsection{Список литература}

	\begin{enumerate}

		\item Т. Кормен, Ч. Лейзерсон, Р. Ривест, К. Штайн. Алгоритмы. Построение и анализ. Издательский дом ``Вильямс'', 2011. 823 - 869.
		\item Б. Страуструп. Язык программирования С++ . Addison-Wesley, 2000. 142 - 149.
		\item Г. Шилдт. С++. Полное руководство. СПб.: Наука и Техника, Издательский дом “Вильямс”, 2006. 621 - 693.
		\item Я. Галовиц. С++17 STL. Стандартная библиотека шаблонов. Серийная библиотека программиста, 2018. 91 - 123.
		\item Р. Седжвик. Фундаментальные алгоритмы С++. Diasoft, 2001. 42 - 69.

	\end{enumerate}

\end{document}
